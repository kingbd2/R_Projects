\documentclass[]{article}
\usepackage{lmodern}
\usepackage{amssymb,amsmath}
\usepackage{ifxetex,ifluatex}
\usepackage{fixltx2e} % provides \textsubscript
\ifnum 0\ifxetex 1\fi\ifluatex 1\fi=0 % if pdftex
  \usepackage[T1]{fontenc}
  \usepackage[utf8]{inputenc}
\else % if luatex or xelatex
  \ifxetex
    \usepackage{mathspec}
  \else
    \usepackage{fontspec}
  \fi
  \defaultfontfeatures{Ligatures=TeX,Scale=MatchLowercase}
\fi
% use upquote if available, for straight quotes in verbatim environments
\IfFileExists{upquote.sty}{\usepackage{upquote}}{}
% use microtype if available
\IfFileExists{microtype.sty}{%
\usepackage{microtype}
\UseMicrotypeSet[protrusion]{basicmath} % disable protrusion for tt fonts
}{}
\usepackage[margin=1in]{geometry}
\usepackage{hyperref}
\hypersetup{unicode=true,
            pdftitle={An Introduction to Statistical Learning - Notes},
            pdfborder={0 0 0},
            breaklinks=true}
\urlstyle{same}  % don't use monospace font for urls
\usepackage{color}
\usepackage{fancyvrb}
\newcommand{\VerbBar}{|}
\newcommand{\VERB}{\Verb[commandchars=\\\{\}]}
\DefineVerbatimEnvironment{Highlighting}{Verbatim}{commandchars=\\\{\}}
% Add ',fontsize=\small' for more characters per line
\usepackage{framed}
\definecolor{shadecolor}{RGB}{248,248,248}
\newenvironment{Shaded}{\begin{snugshade}}{\end{snugshade}}
\newcommand{\KeywordTok}[1]{\textcolor[rgb]{0.13,0.29,0.53}{\textbf{#1}}}
\newcommand{\DataTypeTok}[1]{\textcolor[rgb]{0.13,0.29,0.53}{#1}}
\newcommand{\DecValTok}[1]{\textcolor[rgb]{0.00,0.00,0.81}{#1}}
\newcommand{\BaseNTok}[1]{\textcolor[rgb]{0.00,0.00,0.81}{#1}}
\newcommand{\FloatTok}[1]{\textcolor[rgb]{0.00,0.00,0.81}{#1}}
\newcommand{\ConstantTok}[1]{\textcolor[rgb]{0.00,0.00,0.00}{#1}}
\newcommand{\CharTok}[1]{\textcolor[rgb]{0.31,0.60,0.02}{#1}}
\newcommand{\SpecialCharTok}[1]{\textcolor[rgb]{0.00,0.00,0.00}{#1}}
\newcommand{\StringTok}[1]{\textcolor[rgb]{0.31,0.60,0.02}{#1}}
\newcommand{\VerbatimStringTok}[1]{\textcolor[rgb]{0.31,0.60,0.02}{#1}}
\newcommand{\SpecialStringTok}[1]{\textcolor[rgb]{0.31,0.60,0.02}{#1}}
\newcommand{\ImportTok}[1]{#1}
\newcommand{\CommentTok}[1]{\textcolor[rgb]{0.56,0.35,0.01}{\textit{#1}}}
\newcommand{\DocumentationTok}[1]{\textcolor[rgb]{0.56,0.35,0.01}{\textbf{\textit{#1}}}}
\newcommand{\AnnotationTok}[1]{\textcolor[rgb]{0.56,0.35,0.01}{\textbf{\textit{#1}}}}
\newcommand{\CommentVarTok}[1]{\textcolor[rgb]{0.56,0.35,0.01}{\textbf{\textit{#1}}}}
\newcommand{\OtherTok}[1]{\textcolor[rgb]{0.56,0.35,0.01}{#1}}
\newcommand{\FunctionTok}[1]{\textcolor[rgb]{0.00,0.00,0.00}{#1}}
\newcommand{\VariableTok}[1]{\textcolor[rgb]{0.00,0.00,0.00}{#1}}
\newcommand{\ControlFlowTok}[1]{\textcolor[rgb]{0.13,0.29,0.53}{\textbf{#1}}}
\newcommand{\OperatorTok}[1]{\textcolor[rgb]{0.81,0.36,0.00}{\textbf{#1}}}
\newcommand{\BuiltInTok}[1]{#1}
\newcommand{\ExtensionTok}[1]{#1}
\newcommand{\PreprocessorTok}[1]{\textcolor[rgb]{0.56,0.35,0.01}{\textit{#1}}}
\newcommand{\AttributeTok}[1]{\textcolor[rgb]{0.77,0.63,0.00}{#1}}
\newcommand{\RegionMarkerTok}[1]{#1}
\newcommand{\InformationTok}[1]{\textcolor[rgb]{0.56,0.35,0.01}{\textbf{\textit{#1}}}}
\newcommand{\WarningTok}[1]{\textcolor[rgb]{0.56,0.35,0.01}{\textbf{\textit{#1}}}}
\newcommand{\AlertTok}[1]{\textcolor[rgb]{0.94,0.16,0.16}{#1}}
\newcommand{\ErrorTok}[1]{\textcolor[rgb]{0.64,0.00,0.00}{\textbf{#1}}}
\newcommand{\NormalTok}[1]{#1}
\usepackage{graphicx,grffile}
\makeatletter
\def\maxwidth{\ifdim\Gin@nat@width>\linewidth\linewidth\else\Gin@nat@width\fi}
\def\maxheight{\ifdim\Gin@nat@height>\textheight\textheight\else\Gin@nat@height\fi}
\makeatother
% Scale images if necessary, so that they will not overflow the page
% margins by default, and it is still possible to overwrite the defaults
% using explicit options in \includegraphics[width, height, ...]{}
\setkeys{Gin}{width=\maxwidth,height=\maxheight,keepaspectratio}
\IfFileExists{parskip.sty}{%
\usepackage{parskip}
}{% else
\setlength{\parindent}{0pt}
\setlength{\parskip}{6pt plus 2pt minus 1pt}
}
\setlength{\emergencystretch}{3em}  % prevent overfull lines
\providecommand{\tightlist}{%
  \setlength{\itemsep}{0pt}\setlength{\parskip}{0pt}}
\setcounter{secnumdepth}{0}
% Redefines (sub)paragraphs to behave more like sections
\ifx\paragraph\undefined\else
\let\oldparagraph\paragraph
\renewcommand{\paragraph}[1]{\oldparagraph{#1}\mbox{}}
\fi
\ifx\subparagraph\undefined\else
\let\oldsubparagraph\subparagraph
\renewcommand{\subparagraph}[1]{\oldsubparagraph{#1}\mbox{}}
\fi

%%% Use protect on footnotes to avoid problems with footnotes in titles
\let\rmarkdownfootnote\footnote%
\def\footnote{\protect\rmarkdownfootnote}

%%% Change title format to be more compact
\usepackage{titling}

% Create subtitle command for use in maketitle
\newcommand{\subtitle}[1]{
  \posttitle{
    \begin{center}\large#1\end{center}
    }
}

\setlength{\droptitle}{-2em}

  \title{An Introduction to Statistical Learning - Notes}
    \pretitle{\vspace{\droptitle}\centering\huge}
  \posttitle{\par}
    \author{}
    \preauthor{}\postauthor{}
    \date{}
    \predate{}\postdate{}
  

\begin{document}
\maketitle

\subsection{Chapter 3 - Part A - Linear
Regression}\label{chapter-3---part-a---linear-regression}

In this chapter, we learn about how to fit a simple linear model to
data. The following are the exercises:

\begin{Shaded}
\begin{Highlighting}[]
\KeywordTok{names}\NormalTok{(Boston)}
\end{Highlighting}
\end{Shaded}

\begin{verbatim}
##  [1] "crim"    "zn"      "indus"   "chas"    "nox"     "rm"      "age"    
##  [8] "dis"     "rad"     "tax"     "ptratio" "black"   "lstat"   "medv"
\end{verbatim}

\begin{Shaded}
\begin{Highlighting}[]
\KeywordTok{ggplot}\NormalTok{(Boston, }\KeywordTok{aes}\NormalTok{(}\DataTypeTok{x=}\NormalTok{lstat, }\DataTypeTok{y=}\NormalTok{medv)) }\OperatorTok{+}\StringTok{ }\KeywordTok{geom_point}\NormalTok{() }\OperatorTok{+}\StringTok{ }\KeywordTok{geom_smooth}\NormalTok{(}\DataTypeTok{method=}\NormalTok{lm)}
\end{Highlighting}
\end{Shaded}

\includegraphics{notes_markdown_files/figure-latex/unnamed-chunk-3-1.pdf}

\begin{Shaded}
\begin{Highlighting}[]
\NormalTok{lm.fit=}\KeywordTok{lm}\NormalTok{(medv}\OperatorTok{~}\NormalTok{lstat)}
\KeywordTok{summary}\NormalTok{(lm.fit)}
\end{Highlighting}
\end{Shaded}

\begin{verbatim}
## 
## Call:
## lm(formula = medv ~ lstat)
## 
## Residuals:
##     Min      1Q  Median      3Q     Max 
## -15.168  -3.990  -1.318   2.034  24.500 
## 
## Coefficients:
##             Estimate Std. Error t value Pr(>|t|)    
## (Intercept) 34.55384    0.56263   61.41   <2e-16 ***
## lstat       -0.95005    0.03873  -24.53   <2e-16 ***
## ---
## Signif. codes:  0 '***' 0.001 '**' 0.01 '*' 0.05 '.' 0.1 ' ' 1
## 
## Residual standard error: 6.216 on 504 degrees of freedom
## Multiple R-squared:  0.5441, Adjusted R-squared:  0.5432 
## F-statistic: 601.6 on 1 and 504 DF,  p-value: < 2.2e-16
\end{verbatim}

lm. fit is a linear model fit to the data. The linear model has multiple
attributes, which can be seen and then accessed using the following:

\begin{Shaded}
\begin{Highlighting}[]
\KeywordTok{names}\NormalTok{(lm.fit)}
\end{Highlighting}
\end{Shaded}

\begin{verbatim}
##  [1] "coefficients"  "residuals"     "effects"       "rank"         
##  [5] "fitted.values" "assign"        "qr"            "df.residual"  
##  [9] "xlevels"       "call"          "terms"         "model"
\end{verbatim}

\begin{Shaded}
\begin{Highlighting}[]
\KeywordTok{coef}\NormalTok{(lm.fit)}
\end{Highlighting}
\end{Shaded}

\begin{verbatim}
## (Intercept)       lstat 
##  34.5538409  -0.9500494
\end{verbatim}

Note: coef is a generic function which extracts model coefficients from
objects returned by modeling functions

Can also extract components from the model using ex.:

\begin{Shaded}
\begin{Highlighting}[]
\NormalTok{residuals<-lm.fit}\OperatorTok{$}\NormalTok{residuals}
\CommentTok{# lm.fit$residuals}
\end{Highlighting}
\end{Shaded}

Get \textbf{confidence interval} of coefficient estimates

\begin{Shaded}
\begin{Highlighting}[]
\KeywordTok{confint}\NormalTok{(lm.fit)}
\end{Highlighting}
\end{Shaded}

\begin{verbatim}
##                 2.5 %     97.5 %
## (Intercept) 33.448457 35.6592247
## lstat       -1.026148 -0.8739505
\end{verbatim}

Predict a data value using the predict() function - in this example, we
are predicting the medv, or median house value, at 5\%, 10\%, and 15\%
of houses being low-income. Can produce the confidence interval for the
lstat value of 10 (interval = ``confidence'') or the confidence interval
of the prediction (interval = ``prediction'')

\begin{Shaded}
\begin{Highlighting}[]
\KeywordTok{predict}\NormalTok{(lm.fit, }\KeywordTok{data.frame}\NormalTok{(}\DataTypeTok{lstat=}\KeywordTok{c}\NormalTok{(}\DecValTok{5}\NormalTok{,}\DecValTok{10}\NormalTok{,}\DecValTok{15}\NormalTok{)), }\DataTypeTok{interval =} \StringTok{"confidence"}\NormalTok{)}
\end{Highlighting}
\end{Shaded}

\begin{verbatim}
##        fit      lwr      upr
## 1 29.80359 29.00741 30.59978
## 2 25.05335 24.47413 25.63256
## 3 20.30310 19.73159 20.87461
\end{verbatim}

\begin{Shaded}
\begin{Highlighting}[]
\KeywordTok{predict}\NormalTok{(lm.fit, }\KeywordTok{data.frame}\NormalTok{(}\DataTypeTok{lstat=}\KeywordTok{c}\NormalTok{(}\DecValTok{5}\NormalTok{,}\DecValTok{10}\NormalTok{,}\DecValTok{15}\NormalTok{)), }\DataTypeTok{interval =} \StringTok{"prediction"}\NormalTok{)}
\end{Highlighting}
\end{Shaded}

\begin{verbatim}
##        fit       lwr      upr
## 1 29.80359 17.565675 42.04151
## 2 25.05335 12.827626 37.27907
## 3 20.30310  8.077742 32.52846
\end{verbatim}

Note: can use plot(lstat, medv), abline(lm.fit) to quickly plot data and
line. abline(a,b) plots any line with intercept a, slope b

\begin{Shaded}
\begin{Highlighting}[]
\KeywordTok{plot}\NormalTok{(lstat, medv,}\DataTypeTok{pch=}\DecValTok{20}\NormalTok{)}
\KeywordTok{abline}\NormalTok{(lm.fit)}
\end{Highlighting}
\end{Shaded}

\includegraphics{notes_markdown_files/figure-latex/unnamed-chunk-8-1.pdf}

Plot diagnostic plots:

\begin{Shaded}
\begin{Highlighting}[]
\KeywordTok{par}\NormalTok{(}\DataTypeTok{mfrow=}\KeywordTok{c}\NormalTok{(}\DecValTok{2}\NormalTok{,}\DecValTok{2}\NormalTok{))}
\KeywordTok{plot}\NormalTok{(lm.fit)}
\end{Highlighting}
\end{Shaded}

\includegraphics{notes_markdown_files/figure-latex/unnamed-chunk-9-1.pdf}
Plot residuals using rstudent() function:

\begin{Shaded}
\begin{Highlighting}[]
\KeywordTok{plot}\NormalTok{(}\KeywordTok{predict}\NormalTok{(lm.fit), }\KeywordTok{residuals}\NormalTok{(lm.fit))}
\end{Highlighting}
\end{Shaded}

\includegraphics{notes_markdown_files/figure-latex/unnamed-chunk-10-1.pdf}

\begin{Shaded}
\begin{Highlighting}[]
\KeywordTok{plot}\NormalTok{(}\KeywordTok{predict}\NormalTok{(lm.fit), }\KeywordTok{rstudent}\NormalTok{(lm.fit))}
\end{Highlighting}
\end{Shaded}

\includegraphics{notes_markdown_files/figure-latex/unnamed-chunk-10-2.pdf}

\begin{Shaded}
\begin{Highlighting}[]
\KeywordTok{plot}\NormalTok{(}\KeywordTok{hatvalues}\NormalTok{(lm.fit))}
\end{Highlighting}
\end{Shaded}

\includegraphics{notes_markdown_files/figure-latex/unnamed-chunk-10-3.pdf}

\subsection{Chapter 3 - Part B - Multiple Linear
Regression}\label{chapter-3---part-b---multiple-linear-regression}

\textbf{Multiple linear regression} using lstat and age as predictors

\begin{Shaded}
\begin{Highlighting}[]
\NormalTok{lm.fit=}\KeywordTok{lm}\NormalTok{(medv}\OperatorTok{~}\NormalTok{lstat}\OperatorTok{+}\NormalTok{age, }\DataTypeTok{data=}\NormalTok{Boston)}
\KeywordTok{summary}\NormalTok{(lm.fit)}
\end{Highlighting}
\end{Shaded}

\begin{verbatim}
## 
## Call:
## lm(formula = medv ~ lstat + age, data = Boston)
## 
## Residuals:
##     Min      1Q  Median      3Q     Max 
## -15.981  -3.978  -1.283   1.968  23.158 
## 
## Coefficients:
##             Estimate Std. Error t value Pr(>|t|)    
## (Intercept) 33.22276    0.73085  45.458  < 2e-16 ***
## lstat       -1.03207    0.04819 -21.416  < 2e-16 ***
## age          0.03454    0.01223   2.826  0.00491 ** 
## ---
## Signif. codes:  0 '***' 0.001 '**' 0.01 '*' 0.05 '.' 0.1 ' ' 1
## 
## Residual standard error: 6.173 on 503 degrees of freedom
## Multiple R-squared:  0.5513, Adjusted R-squared:  0.5495 
## F-statistic:   309 on 2 and 503 DF,  p-value: < 2.2e-16
\end{verbatim}

\textbf{Multiple linear regression} using all variables as predictors.

\begin{Shaded}
\begin{Highlighting}[]
\NormalTok{lm.fit=}\KeywordTok{lm}\NormalTok{(medv}\OperatorTok{~}\NormalTok{., }\DataTypeTok{data=}\NormalTok{Boston)}
\KeywordTok{summary}\NormalTok{(lm.fit)}
\end{Highlighting}
\end{Shaded}

\begin{verbatim}
## 
## Call:
## lm(formula = medv ~ ., data = Boston)
## 
## Residuals:
##     Min      1Q  Median      3Q     Max 
## -15.595  -2.730  -0.518   1.777  26.199 
## 
## Coefficients:
##               Estimate Std. Error t value Pr(>|t|)    
## (Intercept)  3.646e+01  5.103e+00   7.144 3.28e-12 ***
## crim        -1.080e-01  3.286e-02  -3.287 0.001087 ** 
## zn           4.642e-02  1.373e-02   3.382 0.000778 ***
## indus        2.056e-02  6.150e-02   0.334 0.738288    
## chas         2.687e+00  8.616e-01   3.118 0.001925 ** 
## nox         -1.777e+01  3.820e+00  -4.651 4.25e-06 ***
## rm           3.810e+00  4.179e-01   9.116  < 2e-16 ***
## age          6.922e-04  1.321e-02   0.052 0.958229    
## dis         -1.476e+00  1.995e-01  -7.398 6.01e-13 ***
## rad          3.060e-01  6.635e-02   4.613 5.07e-06 ***
## tax         -1.233e-02  3.760e-03  -3.280 0.001112 ** 
## ptratio     -9.527e-01  1.308e-01  -7.283 1.31e-12 ***
## black        9.312e-03  2.686e-03   3.467 0.000573 ***
## lstat       -5.248e-01  5.072e-02 -10.347  < 2e-16 ***
## ---
## Signif. codes:  0 '***' 0.001 '**' 0.01 '*' 0.05 '.' 0.1 ' ' 1
## 
## Residual standard error: 4.745 on 492 degrees of freedom
## Multiple R-squared:  0.7406, Adjusted R-squared:  0.7338 
## F-statistic: 108.1 on 13 and 492 DF,  p-value: < 2.2e-16
\end{verbatim}

Get specific values from the model using:

The function summary.lm computes and returns a list of summary
statistics of the fitted linear model given in object, using the
components (list elements) ``call'' and ``terms'' from its argument,
plus:

residuals: the weighted residuals, the usual residuals rescaled by the
square root of the weights specified in the call to lm.

coefficients: a p x 4 matrix with columns for the estimated coefficient,
its standard error, t-statistic and corresponding (two-sided) p-value.
Aliased coefficients are omitted.

aliased: named logical vector showing if the original coefficients are
aliased.

sigma: the square root of the estimated variance of the random error
σ\^{}2 = 1/(n-p) Sum(w{[}i{]} R{[}i{]}\^{}2), where R{[}i{]} is the i-th
residual, residuals{[}i{]}.

df: degrees of freedom, a 3-vector (p, n-p, p*), the first being the
number of non-aliased coefficients, the last being the total number of
coefficients.

fstatistic: (for models including non-intercept terms) a 3-vector with
the value of the F-statistic with its numerator and denominator degrees
of freedom.

r.squared: R\^{}2, the `fraction of variance explained by the model',
R\^{}2 = 1 - Sum(R{[}i{]}\^{}2) / Sum((y{[}i{]}- y\emph{)\^{}2), where
y} is the mean of y{[}i{]} if there is an intercept and zero otherwise.
adj.r.squared the above R\^{}2 statistic `adjusted', penalizing for
higher p.

cov.unscaled: a p x p matrix of (unscaled) covariances of the
coef{[}j{]}, j=1, \ldots{}, p.

correlation: the correlation matrix corresponding to the above
cov.unscaled, if correlation = TRUE is specified.

symbolic.cor: (only if correlation is true.) The value of the argument
symbolic.cor.

na.action: from object, if present there.

EXAMPLE:

\begin{Shaded}
\begin{Highlighting}[]
\KeywordTok{summary}\NormalTok{(lm.fit)}\OperatorTok{$}\NormalTok{r.squared}
\end{Highlighting}
\end{Shaded}

\begin{verbatim}
## [1] 0.7406427
\end{verbatim}

\textbf{Interaction terms} are assessed using X1:X2 notation. Also,
X1*X2 includes X1, X2, and X1:X2 as predictors. For example:

\begin{Shaded}
\begin{Highlighting}[]
\NormalTok{lm.fit<-}\KeywordTok{lm}\NormalTok{(medv}\OperatorTok{~}\NormalTok{lstat}\OperatorTok{*}\NormalTok{age, }\DataTypeTok{data =}\NormalTok{ Boston)}
\KeywordTok{summary}\NormalTok{(lm.fit)}
\end{Highlighting}
\end{Shaded}

\begin{verbatim}
## 
## Call:
## lm(formula = medv ~ lstat * age, data = Boston)
## 
## Residuals:
##     Min      1Q  Median      3Q     Max 
## -15.806  -4.045  -1.333   2.085  27.552 
## 
## Coefficients:
##               Estimate Std. Error t value Pr(>|t|)    
## (Intercept) 36.0885359  1.4698355  24.553  < 2e-16 ***
## lstat       -1.3921168  0.1674555  -8.313 8.78e-16 ***
## age         -0.0007209  0.0198792  -0.036   0.9711    
## lstat:age    0.0041560  0.0018518   2.244   0.0252 *  
## ---
## Signif. codes:  0 '***' 0.001 '**' 0.01 '*' 0.05 '.' 0.1 ' ' 1
## 
## Residual standard error: 6.149 on 502 degrees of freedom
## Multiple R-squared:  0.5557, Adjusted R-squared:  0.5531 
## F-statistic: 209.3 on 3 and 502 DF,  p-value: < 2.2e-16
\end{verbatim}

\begin{Shaded}
\begin{Highlighting}[]
\KeywordTok{summary}\NormalTok{(lm.fit)}\OperatorTok{$}\NormalTok{r.squared}
\end{Highlighting}
\end{Shaded}

\begin{verbatim}
## [1] 0.5557265
\end{verbatim}

\textbf{Non-linear transformation} can be done using I(X\^{}2). For
example:

\begin{Shaded}
\begin{Highlighting}[]
\NormalTok{lm.fit<-}\KeywordTok{lm}\NormalTok{(medv}\OperatorTok{~}\NormalTok{lstat}\OperatorTok{+}\KeywordTok{I}\NormalTok{(lstat}\OperatorTok{^}\DecValTok{2}\NormalTok{), }\DataTypeTok{data =}\NormalTok{ Boston)}
\KeywordTok{summary}\NormalTok{(lm.fit)}
\end{Highlighting}
\end{Shaded}

\begin{verbatim}
## 
## Call:
## lm(formula = medv ~ lstat + I(lstat^2), data = Boston)
## 
## Residuals:
##      Min       1Q   Median       3Q      Max 
## -15.2834  -3.8313  -0.5295   2.3095  25.4148 
## 
## Coefficients:
##              Estimate Std. Error t value Pr(>|t|)    
## (Intercept) 42.862007   0.872084   49.15   <2e-16 ***
## lstat       -2.332821   0.123803  -18.84   <2e-16 ***
## I(lstat^2)   0.043547   0.003745   11.63   <2e-16 ***
## ---
## Signif. codes:  0 '***' 0.001 '**' 0.01 '*' 0.05 '.' 0.1 ' ' 1
## 
## Residual standard error: 5.524 on 503 degrees of freedom
## Multiple R-squared:  0.6407, Adjusted R-squared:  0.6393 
## F-statistic: 448.5 on 2 and 503 DF,  p-value: < 2.2e-16
\end{verbatim}

\begin{Shaded}
\begin{Highlighting}[]
\KeywordTok{summary}\NormalTok{(lm.fit)}\OperatorTok{$}\NormalTok{r.squared}
\end{Highlighting}
\end{Shaded}

\begin{verbatim}
## [1] 0.6407169
\end{verbatim}

Compare two models using ANOVA. Below, comparing the two linear models,
one with and one without the quadratic term.

\begin{Shaded}
\begin{Highlighting}[]
\NormalTok{lm.fit1=}\KeywordTok{lm}\NormalTok{(medv}\OperatorTok{~}\NormalTok{lstat, }\DataTypeTok{data =}\NormalTok{ Boston)}
\NormalTok{lm.fit2=}\KeywordTok{lm}\NormalTok{(medv}\OperatorTok{~}\NormalTok{lstat}\OperatorTok{+}\KeywordTok{I}\NormalTok{(lstat}\OperatorTok{^}\DecValTok{2}\NormalTok{), }\DataTypeTok{data =}\NormalTok{ Boston)}
\KeywordTok{anova}\NormalTok{(lm.fit1, lm.fit2)}
\end{Highlighting}
\end{Shaded}

\begin{verbatim}
## Analysis of Variance Table
## 
## Model 1: medv ~ lstat
## Model 2: medv ~ lstat + I(lstat^2)
##   Res.Df   RSS Df Sum of Sq     F    Pr(>F)    
## 1    504 19472                                 
## 2    503 15347  1    4125.1 135.2 < 2.2e-16 ***
## ---
## Signif. codes:  0 '***' 0.001 '**' 0.01 '*' 0.05 '.' 0.1 ' ' 1
\end{verbatim}

\begin{Shaded}
\begin{Highlighting}[]
\KeywordTok{par}\NormalTok{(}\DataTypeTok{mfrow=}\KeywordTok{c}\NormalTok{(}\DecValTok{2}\NormalTok{,}\DecValTok{2}\NormalTok{))}
\KeywordTok{plot}\NormalTok{(lm.fit2)}
\end{Highlighting}
\end{Shaded}

\includegraphics{notes_markdown_files/figure-latex/unnamed-chunk-16-1.pdf}

\begin{Shaded}
\begin{Highlighting}[]
\KeywordTok{plot}\NormalTok{(lstat, medv,}\DataTypeTok{pch=}\DecValTok{20}\NormalTok{)}
\KeywordTok{abline}\NormalTok{(lm.fit2)}
\end{Highlighting}
\end{Shaded}

\begin{verbatim}
## Warning in abline(lm.fit2): only using the first two of 3 regression
## coefficients
\end{verbatim}

\includegraphics{notes_markdown_files/figure-latex/unnamed-chunk-16-2.pdf}

\subsubsection{Qualitative Predictors}\label{qualitative-predictors}

Example using the Carseats dataset from the ISLR package:

\begin{Shaded}
\begin{Highlighting}[]
\KeywordTok{names}\NormalTok{(ISLR}\OperatorTok{::}\NormalTok{Carseats)}
\end{Highlighting}
\end{Shaded}

\begin{verbatim}
##  [1] "Sales"       "CompPrice"   "Income"      "Advertising" "Population" 
##  [6] "Price"       "ShelveLoc"   "Age"         "Education"   "Urban"      
## [11] "US"
\end{verbatim}

\begin{Shaded}
\begin{Highlighting}[]
\NormalTok{lm.fit=}\KeywordTok{lm}\NormalTok{(Sales}\OperatorTok{~}\NormalTok{.}\OperatorTok{+}\NormalTok{Income}\OperatorTok{:}\NormalTok{Advertising}\OperatorTok{+}\NormalTok{Price}\OperatorTok{:}\NormalTok{Age,}\DataTypeTok{data=}\NormalTok{ISLR}\OperatorTok{::}\NormalTok{Carseats)}
\KeywordTok{summary}\NormalTok{(lm.fit)}
\end{Highlighting}
\end{Shaded}

\begin{verbatim}
## 
## Call:
## lm(formula = Sales ~ . + Income:Advertising + Price:Age, data = ISLR::Carseats)
## 
## Residuals:
##     Min      1Q  Median      3Q     Max 
## -2.9208 -0.7503  0.0177  0.6754  3.3413 
## 
## Coefficients:
##                      Estimate Std. Error t value Pr(>|t|)    
## (Intercept)         6.5755654  1.0087470   6.519 2.22e-10 ***
## CompPrice           0.0929371  0.0041183  22.567  < 2e-16 ***
## Income              0.0108940  0.0026044   4.183 3.57e-05 ***
## Advertising         0.0702462  0.0226091   3.107 0.002030 ** 
## Population          0.0001592  0.0003679   0.433 0.665330    
## Price              -0.1008064  0.0074399 -13.549  < 2e-16 ***
## ShelveLocGood       4.8486762  0.1528378  31.724  < 2e-16 ***
## ShelveLocMedium     1.9532620  0.1257682  15.531  < 2e-16 ***
## Age                -0.0579466  0.0159506  -3.633 0.000318 ***
## Education          -0.0208525  0.0196131  -1.063 0.288361    
## UrbanYes            0.1401597  0.1124019   1.247 0.213171    
## USYes              -0.1575571  0.1489234  -1.058 0.290729    
## Income:Advertising  0.0007510  0.0002784   2.698 0.007290 ** 
## Price:Age           0.0001068  0.0001333   0.801 0.423812    
## ---
## Signif. codes:  0 '***' 0.001 '**' 0.01 '*' 0.05 '.' 0.1 ' ' 1
## 
## Residual standard error: 1.011 on 386 degrees of freedom
## Multiple R-squared:  0.8761, Adjusted R-squared:  0.8719 
## F-statistic:   210 on 13 and 386 DF,  p-value: < 2.2e-16
\end{verbatim}

The ShelveLoc predictor is a categorical variable that gets transformed
into dummy variables. We can see the dummy variables using contrasts:

\begin{Shaded}
\begin{Highlighting}[]
\KeywordTok{attach}\NormalTok{(ISLR}\OperatorTok{::}\NormalTok{Carseats)}
\KeywordTok{contrasts}\NormalTok{(ShelveLoc)}
\end{Highlighting}
\end{Shaded}

\begin{verbatim}
##        Good Medium
## Bad       0      0
## Good      1      0
## Medium    0      1
\end{verbatim}


\end{document}
